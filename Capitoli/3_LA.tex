\chapter{Learning Analytics}

\section{Introduzione}

\subsection{Definizione}

Nel primo capitolo è stato accennato che i dati raccolti con l'uso dei Learning Management System vengono analizzati con tecniche di Learning Analytics, ma cosa sono esattamente queste tecniche?

I Learning Management System sono tecnologie nuove, di conseguenza anche il Learning Analytics è un'area di studio nuova, proprio per questo non esiste ancora una definizione universale. \\
Quella più popolare è quella adottata dalla Society for Learning Analytics Research (SoLAR) \cite{Solaresearch} :

"Con Learning Analytics ci si riferisce alla misurazione, alla raccolta, all’analisi e alla presentazione dei dati sugli studenti e sui loro contesti, ai fini della comprensione e dell’ottimizzazione dell’apprendimento e degli ambienti in cui ha luogo." \cite{first}

In sostanza, con questo termine si indica la disciplina che si occupa di raccogliere, analizzare e comunicare i dati riguardanti gli studenti, allo scopo di comprendere e quindi migliorare la didattica.

\clearpage

\subsection{Discipline collegate}

Il concetto di Learning Analytics si mischia e spesso di confonde con altre 2 materie:

\begin{itemize}

\item L’Academic Analytics:

Si focalizza sul migliorare le opportunità di apprendimento e i risultati scolastici a livello nazionale e internazionale.

Più che sui singoli studenti e sui corsi di una scuola, questa disciplina si occupa principalmente di analizzare l'intera struttura scolastica in generale.

Confrontare le varie università di una particolare area geografica, per scoprire qual è la migliore, è un'applicazione di questa disciplina.

\cite{academicanalytics}

\item Educational Data Mining:

Branca del Data Mining, una tecnologia che si occupa inizialmente dell'estrazione di un'enorme quantità di dati da sistemi informatici.

In seguito, le informazioni cosi ottenute vengono analizzate principalmente in 3 diversi modi:

\begin{itemize}

\item Esplorazione: Si cerca di riassumere i dati.
\item Classificazione E Clustering: Si dividono gli elementi in gruppi simili.

Nella classificazione i gruppi sono scelti prima di effettuare l'analisi, \\ nel clustering, invece, questi vengono automaticamente creati.

\end{itemize}

L' Educational Data Mining quindi si occupa di applicare le tecniche di Data Mining ai dati che riguardano l'educazione degli studenti.

\cite{educationaldatamining}

\end{itemize}

Come si può vedere dalle definizioni, queste 3 aree sono strettamente collegate tra di loro e molto spesso si sovrappongono.

Ad esempio, nel corso di Learning Analytics offerto dalla "Teachers College" della "Columbia University" gli studenti apprendono anche tecniche di Educational Data Mining. \cite{columbia}

\clearpage

\section{Tecniche}

Viene presentato un elenco delle principali metodologie di Learning Analytics.

\subsection{Monitoraggio}
Si osservano gli studenti per tutto la durata del corso.

In particolare, si monitora come, quanto, quando e dove essi studiano.

\subsection{Divisione degli studenti in gruppi, personalizzazione del percorso di studi, predizione del futuro}

\subsubsection{Divisione}

Un esempio di divisione in gruppi potrebbe essere il seguente:

\begin{enumerate}

\item Studenti che non studiano abbastanza. 
\item Studenti che stanno avendo problemi a capire i concetti delle lezioni.
\item Studenti che non stanno avendo problemi.

\end{enumerate}

A questo punto, il docente può:

\begin{itemize}

\item Cercare di fare capire l'utilità dell'uso frequente del materiale agli studenti del primo gruppo.
\item Provare a capire e risolvere i problemi degli studenti della seconda categoria.
\item Fare i complimenti agli studenti del terzo insieme, incoraggiandoli a continuare così.

\end{itemize}

Poiché i gruppi cambiano con il passare del tempo, il professore è in grado di valutare l'efficacia delle sue azioni.

Ad esempio, se uno studente passa dal primo gruppo al terzo, allora il suo intervento è stato positivo.

\subsubsection{Personalizzazione}

I contenuti del corso possono anche cambiare in base alla divisione effettuata.

Ad esempio, mentre gli studenti del secondo gruppo vedranno materiale contenente i concetti principali, quelli del terzo vedranno contenuti aggiuntivi di approfondimento.

Si ottiene così un'esperienza di studio personalizzata.

\subsubsection{Predizione}

Il sistema di Learning Analytics, con la divisione in gruppi, è in grado di prevedere quali studenti avranno difficolta in futuro e quali supereranno il corso senza problemi.

I software più avanzati sono addirittura in grado di predire il voto che lo studente prenderà all' esame. 

\subsection{Competizione tra studenti}

Per tutta la durata del corso, lo studente può osservare come studiano gli altri allievi.

Il docente può anche organizzare competizioni con premi per incentivare lo studio.

\subsection{Analisi sull'utilizzo del materiale}

Al termine del corso, il docente può confrontare come è stato utilizzato il materiale proposto agli studenti.

Migliorando il materiale poco usato, la qualità della didattica migliorerà per le edizioni successive del corso.

\subsection{Efficacia dello studio in confronto al voto finale}

Al termine del corso, il docente può confrontare il voto preso dallo studente all'esame con il modo in cui ha studiato.

Così, si può scoprire quanto l'utilizzo del materiale sia stato utile per gli studenti.

\section{Sviluppo Di Un Sistema Learning Analytics}

Il processo di Learning Analytics si svolge in 3 fasi:

\begin{enumerate}

\item Raccolta dei dati ottenuti dall'utilizzo dei Learning Management Systems.
\item Analisi.
\item Comunicazione:

I risultati delle analisi vengono presentati, grazie all'utilizzo di grafici, tabelle ed immagini.

\end{enumerate}

Queste 3 frasi si svolgono in maniera ciclica.

Quando il sistema rileva un cambiamento dei dati, dovuto all'interazione dello studente con il corso, si ritorna alla prima fase.

In questo modo, le informazioni comunicate sono sempre aggiornate in tempo reale.

\cite{iadlearning1}

\section{Beneficiari}

Le tecniche di Learning Analytics favoriscono principalmente gli studenti e i docenti.

I primi ottengo:

\begin{itemize}

\item Un percorso di studio personalizzato.
\item Degli aiuti mirati in base alle loro esigenze.    
\item Un miglioramento delle loro performance.

\end{itemize}

I secondi ricevono:

\begin{itemize}

\item Un continuo miglioramento del materiale del corso.
\item La soddisfazione nel vedere i propri studenti arrivare al successo negli studi.

\cite{iadlearning2}

\end{itemize}

\section{Limiti E Complicazioni}

\subsection{Accuratezza dei dati e dei risultati}

Bisogna assicurarsi che i dati iniziali e i risultati ottenuti dopo l'analisi siano corretti. 
Infatti, degli errori di registrazione dell'attività di uno studente, o degli sbagli nella fase di analisi, causano problemi.

Ad esempio, uno studente con difficolta potrebbe essere catalogato erroneamente fra gli studenti che stanno andando bene. \\
Così, il docente non riuscirebbe ad intervenire correttamente nei confronti di questo allievo.

\subsection{Facilità di presentazione}

Le informazioni ottenute dopo l'analisi devono essere comunicate in maniera chiara.

Inoltre, l'utente che utilizza il sistema dovrebbe essere in grado di capire intuitivamente i dati che sta visualizzando.

\subsection{Consenso e privacy}

Innanzitutto, bisogna assicurarsi che tutti gli studenti siano informati e che diano il consenso al raccoglimento dei loro dati riguardanti il modo in cui studiano.

Inoltre, è necessario che queste informazioni vengano in qualche modo rese anonime.

Per finire il software di Learning Analytics deve essere sicuro, solo le persone autorizzate devono avere accesso e i dati non devono essere rubati da malintenzionati.

\subsection{Mancanza di standard}

Poiché il Learning Analytics è un campo di studi ancora nuovo, non è ancora stato definito uno standard comune.

Inoltre, non esistono né strumenti universali per ottenere i dati, né per analizzarli e tanto meno per visualizzarli correttamente.

\subsection{Difficolta nell' ottenere altri dati}

Il sistema di Learning Analytics può analizzare solo come lo studente utilizza il corso online, ma non conosce nulla riguardo lo studio dell'allievo fuori dal questo ambiente virtuale.

Ad esempio, un utente che per qualsiasi motivo decide di non usare il materiale del corso e preferisce studiare tradizionalmente, verrà comunque contato erroneamente tra gli studenti inattivi.

\cite{gradiant}

\section{Esempi}

\subsection{StREAM}

Acronimo di Student Retention Engagement Attainment Monitorning, StREAM \cite{solutionpath} è un software di Learning Analytics che si concentra sui corsi universitari.

Monitora giornalmente gli studenti per osservare il loro progresso nel corso del tempo ed individuare quelli a rischio fallimento.

Il sistema è anche in grado di comprendere le abitudini degli studenti ed identificare quando un allievo cambia drasticamente il suo modo di studiare.

Tutte queste informazioni vengono comunicate tramite dashboard personalizzabili.

\subsection{SNAPP}

Acronimo di Social Networks Adapting Pedagogical Practice.

SNAPP \cite{Snapp} studia come gli studenti interagiscono tra di loro nei forum dei principali Learning Management Systems.

Al termine di questa analisi viene creato un diagramma, che mostra le connessioni fra gli allievi. \\ In questo modo si identificano gli studenti più partecipi e quelli più isolati, ovvero quelli a rischio di fallimento.

Inoltre, questo software è in grado di capire l'oggetto dei messaggi scambiati fra gli alunni. Gli argomenti più discussi, probabilmente, sono anche quelli dove gli studenti stanno avendo una maggiore difficolta.

\subsection{Intelliboard}

Intelliboard è uno dei software di Learning Analytics più famosi ed utilizzati, infatti vanta, secondo il sito ufficiale consultato a Marzo 2019, più di un milione di corsi e 16 milioni di studenti analizzati.

Per ogni corso questa tecnologia offre delle dashboard e dei report: 

\begin{itemize}
\item Le dashboard sono formate principalmente da grafici ed immagini che servono a mostrare dati in modo sintetico. \\ Ogni dashboard è personalizzabile, ovvero si può scegliere quali elementi mostrare e quali no.
\item I report sono formati principalmente da tabelle che mostrano le informazioni in modo dettagliato. \\ Si possono anche applicare particolari filtri per selezionare solo un certo tipo di dati.
\end{itemize}

Intelliboard è un software che presenta numerose funzionalità, verranno qui elencate solo quelle più rilevanti:

\begin{itemize}

\item Monitoraggio del progresso degli studenti, con individuazioni di quelli a rischio fallimento.
\item Analisi del materiale più utilizzato e della partecipazione ai forum.
\item Approfondimento sull'uso dei quiz.
\item Confronto fra gli alunni.

\end{itemize}

Inoltre, permette di comparare dati anche fra corsi diversi. \\ Ad esempio, in questo modo si può conoscere quali sono quelli più seguiti dagli studenti.

Un altro punto di forza è il fatto che Intelliboard supporta i più popolari LMS, tra cui Moodle, BlackBoard e Canvas.

L'unico svantaggio di questo software potrebbe essere il costo troppo eccessivo. \\ Infatti, i prezzi variano dai 588 ai 9500 dollari all’anno, a seconda delle funzionalità che si desidera utilizzare e dal numero di studenti supportati. \\ Fortunatamente esiste anche una versione totalmente gratuita ma molto limitata e una versione di prova della durata di 14 giorni.

\cite{Intelliboard}