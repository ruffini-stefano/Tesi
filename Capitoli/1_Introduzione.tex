\mainmatter

\chapter{Introduzione}

Lo sviluppo tecnologico e l'avvento di Internet hanno portato ad un cambiamento nella didattica.

All'approccio tradizionale delle lezioni in classe, si è affiancato, ed in alcuni casi addirittura sostituito, quello chiamato Apprendimento Online. \\
Quest'ultimo prevede l'utilizzo di dispositivi informatici e della connessione in rete per visualizzare ed interagire con il materiale scolastico.

I Learning Management Systems sono software che offrono, al docente che gestisce l'insegnamento, uno spazio virtuale dove poter caricare tutto ciò che potrebbe essere utile agli studenti. \\
Inoltre, questi strumenti memorizzano dei dati ogni volta che qualcuno utilizza del materiale caricato sulla piattaforma. 

Queste informazioni verranno analizzate con tecniche di Learning Analytics, ovvero, la disciplina che si occupa di studiare come gli allievi apprendono, per migliorare la didattica.

In questo elaborato verranno descritti questi due concetti e verrà presentato un esempio dell'applicazione di questi argomenti.
