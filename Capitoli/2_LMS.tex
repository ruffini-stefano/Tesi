\chapter{Learning Management Systems}

\section{Introduzione}

Negli ultimi anni sempre più scuole, università e in generale tutte le istituzioni che si occupano di formazione, hanno rivoluzionato la loro didattica, usando particolari software chiamati Learning Management Systems.

Un Learning Management System, o più semplicemente LMS, offre al docente uno spazio online dove poter caricare, per ogni corso insegnato, tutto il materiale utile agli studenti, tra cui:

\begin{itemize}

\item Appunti e registrazioni audio/video delle lezioni.
\item Quiz con domande a risposte aperte e chiuse che gli studenti possono svolgere per valutare la loro preparazione.
\item Compiti che gli studenti devono eseguire entro una certa data.
\item Simulazioni degli esami.
\item Link ad alto materiale di approfondimento.
\item Avvisi che riguardano il corso.

\end{itemize}

Oltre ad interagire con il materiale, gli studenti possono porre domande al docente e contattarlo per qualsiasi evenienza.

\cite{Talentlms}

\clearpage

\section{Vantaggi}

I vantaggi di questo approccio sono numerosi, infatti lo studente può:

\begin{itemize}

\item Confrontare i propri appunti con quelli del professore.
\item Riascoltare/Rivedere i concetti che li risultano poco chiari o l'intera lezione in caso l'abbia persa.
\item Verificare la propria preparazione con quiz, compiti e simulazioni.
\item Porre domande al docente in qualsiasi momento.
\item Rimanere sempre aggiornato sulla organizzazione del corso.
\item Studiare dove e quando vuole.

\end{itemize}

Ogni volta che lo studente interagisce con i contenuti del corso, un numero che lo identifica univocamente, data, ora e l'azione compiuta vengono salvati dal sistema. \\
I dati cosi ottenuti saranno poi analizzati con tecniche di Learning Analytics.

\cite{Elearningindustry}

\section{Svantaggi}

Uno studente che usa esclusivamente questi software potrebbe sentirsi:

\begin{itemize}

\item Disorientato dalla quantità del materiale o dal suo uso.
\item Isolato dalla mancata interazione fisica con il docente o con altri studenti.
\item Poco motivato nello studio.  

\end{itemize}

Allo stesso tempo, il docente potrebbe perdere il feedback visivo da parte degli studenti e quindi avere difficolta a capire se i suoi allievi 
siano interessati agli argomenti trattati oppure stiano avendo problemi.

\cite{Joomlalms}

\section{Esempi}

Un esempio di Learning Management System è Moodle \cite{moodle}, usato anche dall'università Bicocca.

Questo software, creato per la prima volta nel 2001 da Martin Dougiamas, usando PHP e JavaScript, gestisce più di 18 milioni di corsi, secondo il sito ufficiale consultato a Marzo 2019.

Moodle è: 

\begin{itemize}
\item Modulare: le sue funzionalità sono disponibili in moduli, che possono essere aggiunti e tolti a piacimento. In questo modo un corso su Moodle sarà diverso da un altro, a seconda delle esigenze del docente.
\item Open source: il suo codice sorgente è disponibile a chiunque, in questo modo gli sviluppatori software possono creare nuovi moduli.
\end{itemize}

Altri esempi di Learning Management System sono Blackboard \cite{Blackboard} e Canvas \cite{Canvas}.
