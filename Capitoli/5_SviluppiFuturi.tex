\chapter{Sviluppi Futuri}

La Dashboard sviluppata vuole essere solo un esempio di applicazione delle tecniche di Learning Analytics. 

Ecco come potrebbe essere migliorata in futuro:
\begin{itemize}
\item Al posto di estrarre i dati a fine corso, sarebbe utile averli in tempo reale, ovvero aggiornati ogni volta che uno studente compie un'attività. \\ In questo modo, potrebbero essere aggiunte queste nuove funzionalità:
\begin{itemize}
\item Identificazione degli allievi a rischio di fallimento con il successivo intervento del docente.
\item Previsione della valutazione che l'alunno prenderà all’esame, in base al come sta attualmente studiando e ai dati delle edizioni precedenti del corso. 
\end{itemize}
\item Miglioramento del modo in cui gli studenti sono divisi in gruppi. \\ Oltre ad analizzare i quiz che hanno completato, potrebbe essere utile prendere in considerazione anche quante volte e quando sono stati terminati. 
\item Creazione di un sistema di login, in grado di classificare gli utenti in docente e studenti. \\ Una volta entrati nella piattaforma, verrà mostrata la corretta dashboard, a seconda della tipologia di utente.
\item Rendere il sito responsive, ovvero facilmente visibile su tutti gli schermi e su tutti i dispositivi.
\item Infine, si potrebbe rilasciare una prima versione del software ad un gruppo di studenti, per provare le sue funzionalità e testarne l'utilità.
\end{itemize}

Per concludere, con la complicità dei docenti, la piattaforma potrebbe essere utilizzata anche per gli altri corsi tenuti dall'Ateneo.